% \iffalse meta-comment
% Copyright (C) 2022
% by Phelype Oleinik <phelype.oleinik@latex-project.org>
% Paulo Cereda <cereda.paulo@gmail.com>
% and samcarter <sam.carter8@aol.com>
% 
% This work may be distributed and/or modified under the conditions of
% the LaTeX Project Public License, either version 1.3c of this license
% or (at your option) any later version. The latest version of this
% license is in
% 
%     http://www.latex-project.org/lppl.txt
% 
% and version 1.3c or later is part of all distributions of LaTeX
% version 2005/12/01 or later.
% 
% This work is "maintained" (as per the LPPL maintenance status)
% by Phelype Oleinik and Paulo Cereda.
% 
%<*package>
\def\sillypageDate{2022-01-20}
\def\sillypageVersion{1.3}
%</package>
%<*driver>
\documentclass[a4paper,full]{l3doc}
\usepackage[T1]{fontenc}
\usepackage{tgbonum}
\usepackage[scaled=.85]{beramono}
\usepackage{sillypage}
\usepackage{hyperref}
\begin{document}
  \DocInput{\jobname.dtx}
\end{document}
%</driver>
% \fi
%
% \GetFileInfo{\jobname.sty}
%
% \title{^^A
%   The \pkg{sillypage} package^^A
%   \thanks{This file has version number \sillypageVersion,
%           last revised \sillypageDate.}
% }
%
% \author{^^A
%  Phelype \emph{lemon curry} Oleinik\thanks
%    {^^A
%      E-mail:
%        \href{mailto:phelype.oleinik@latex-project.org}
%          {phelype.oleinik\meta{at}latex-project.org}^^A
%    }\\
%  Paulo \emph{dictates a letter} Cereda\thanks
%    {^^A
%      E-mail:
%        \href{mailto:cereda.paulo@gmail.com}
%          {cereda.paulo\meta{at}gmail.com}^^A
%    }\\
%  samcarter \emph{cheesy comestibles}\thanks
%    {^^A
%     E-mail:
%        \href{mailto:sam.carter8@aol.com}
%          {sam.carter8\meta{at}aol.com}^^A
%    }^^A
% }
%
% \date{Released \sillypageDate}
% \pagenumbering{silly}
%
% \maketitle
%
% \begin{documentation}
%
% \section{Introduction}
%
% This package is inspired by \emph{The Ministry of Silly Walks}, a famous
% sketch from Monty Python. According to the corresponding
% \href{https://bit.ly/33jKsl1}{Wikipedia article}:
%
% \begin{quotation}
% The sketch as originally depicted in the series begins with John Cleese
% playing Mr.~Teabag, a civil servant who, after purchasing The Times
% from the newsagent in the previous sketch, walks through the streets
% of London (at the crossing of Thorpebank Road and Dunraven Road) in a
% very peculiar manner. He eventually arrives at his place of business:
% The Ministry of Silly Walks, on the northern end of Whitehall. In the
% hallway, he passes other employees all exhibiting their own silly walks
% before arriving at his office. Once there, he finds Mr.~Putey (Michael
% Palin) waiting for him and apologizes for the delay, explaining that
% his walk has become particularly silly of late and it takes longer for
% him to reach his destination.
% \end{quotation}
%
% So we decided to bring this masterpiece of British pop culture to
% the \TeX{} world through a silly page numbering style. Make sure to
% use this package in your thesis, it will be a success.
%
% \section{Usage}
%
% Using a silly package like this is quite straighforward! For starters,
% make sure to include it in your document preamble through
%
% \begin{verbatim}
% \usepackage{sillypage}
% \end{verbatim}
%
% \noindent and that is it! The available interface is as follows:
%
% \begin{function}{\pagenumbering{silly}}
%   \begin{syntax}
%     \cs{pagenumbering}\{silly\}
%   \end{syntax}
%   Write \cs{pagenumbering}|{silly}| in you document to use the page
%   numbering style defined by this package. According to Wikipedia,
%   the typical silly walk gait consists of twelve specific steps to
%   be performed, so page numbering will cycle through these steps,
%   in an kineograph fashion.
% \end{function}
%
% \begin{function}{\silly}
%   \begin{syntax}
%     \cs{silly} \Arg{counter}
%   \end{syntax}
%   Use the macro \cs{silly} on a \LaTeX{} \meta{counter} to typeset the
%   corresponding image for the value of that counter. As noted above,
%   the value will be mapped to an image in the silly walk cycle. Also,
%   observe that this macro is applied to counters and not to integers.
% \end{function}
%
% \begin{function}{\sillystep}
%   \begin{syntax}
%     \cs{sillystep} \Arg{integer}
%   \end{syntax}
%   Prints the step number \meta{integer} from the sequence of steps.
%   This macro works exactly like \cs{silly}, but on integers instead
%   of counters. The integer value will be mapped to an image in the
%   silly walk cycle.
% \end{function}
%
% \bigskip
%
% What a silly yet marvellous interface!
%
% \section{A sample document}
%
% As a means to demonstrate how straightforward (and of course silly) our package
% is, here is a minimal working example:
%
% \begin{verbatim}
% \documentclass{article}
% 
% \usepackage{sillypage}
% \usepackage{blindtext}
% 
% \begin{document}
% \pagenumbering{silly}
% \Blinddocument
% \end{document}
% \end{verbatim}
%
% After compiling it, take a look at the page numbering: you will be greeted with
% the silly walk routine! To impress your supervisor, use our package in your thesis,
% print it and ask them to view those pages in quick succession, so the images appear
% to animate by simulating motion! It is known to work with thesis committees as well.
%
% \section{The silly walk routine}
%
% This is the complete silly walk routine as described by Jazeen Hollings in the
% \href{https://bit.ly/3fc9x46}{corresponding Wikipedia article}. Please note that the
% images used in this package are created by samcarter and released under
% the \href{https://creativecommons.org/licenses/by-sa/4.0/}{CC BY-SA 4.0} license.
% Image sources (vector format) are available in the
% \href{https://github.com/cereda/sillypage}{package repository}
%
% \bigskip
%
% \makeatletter
% {\centering
%  \def\do#1{\ifnum#1>\silly@steps\else
%    \sillystep{#1}
%    \expandafter\do\expandafter{%^^A
%      \number\numexpr#1+1\relax}\fi}\do{1}\par}
% \makeatother
%
% \bigskip
%
% We also recommend Pilates.
%
% \end{documentation}
%
% \begin{implementation}
%
% \section{Implementation details}
%
% The implementation is quite straightforward. Feel free to contact us for
% suggestions and criticism, code improvements and of course other silly
% walk variants.
%
%    \begin{macrocode}
%<*package>
%<@@=silly>
%    \end{macrocode}
%
% \subsection{Package requirements}
%
%    \begin{macrocode}
\ProvidesExplPackage
  {sillypage} {\sillypageDate} {\sillypageVersion}
  {The silly walk as page numbering style}
\RequirePackage{graphicx}
%    \end{macrocode}
%
% \subsection{Picture-specific macros}
%
% \begin{macro}[int]{
%      \silly@steps,
%      \silly@picture,
%      \silly@height}
%    \begin{macrocode}
\newcommand*\silly@steps{12}
\newcommand*\silly@height{.8cm}
\newcommand*\silly@picture{sillywalk-map.pdf}
%    \end{macrocode}
% \end{macro}
%
% \subsection{Single step macros}
%
% \begin{macro}{\silly,\sillystep}
% \begin{macro}[int]{\@silly}
% \begin{macro}{\@@_step:n}
%    \begin{macrocode}
\NewDocumentCommand \silly { m } { \exp_args:Nv \@@_step:n { c@#1 } }
\NewDocumentCommand \sillystep { m } { \@@_step:n {#1} }
\newcommand*\@silly { \@@_step:n }
\cs_new:Npn \@@_step:n #1
  {
    \@@_step_aux:eN
      { \int_eval:n { \int_mod:nn { #1-1 } { \silly@steps } +1 } }
      \c_false_bool
  }
%    \end{macrocode}
% \end{macro}
% \end{macro}
% \end{macro}
%
% \subsection{Multi step macros}
%
% \begin{macro}{\sillynumeral}
% \begin{macro}[int]{\@sillynumeral}
% \begin{macro}{\@@_numeral:n,\@@_numeral_aux:n}
%    \begin{macrocode}
\NewDocumentCommand \sillynumeral { m } { \@@_numeral:n {#1} }
\newcommand*\@sillynumeral { \@@_numeral:n }
\cs_new:Npn \@@_numeral:n #1
  {
    \exp_args:Ne \str_map_function:nN
      { \int_to_base:nn {#1} { 12 } } \@@_numeral_aux:n
  }
\cs_new:Npn \@@_numeral_aux:n #1
  { \@@_step_aux:eN { "\char_uppercase:N #1+1 } \c_true_bool }
%    \end{macrocode}
% \end{macro}
% \end{macro}
% \end{macro}
%
% \subsection{Internals}
%
% \begin{macro}{\@@_step_aux:nN,\@@_step_aux:eN,\@@_trim_step:nN}
%    \begin{macrocode}
\cs_new:Npn \@@_step_aux:nN #1 #2
  {
    \cs_if_exist_use:cF { texorpdfstring } { \use_i:nn }
      { \@@_trim_step:nN {#1} #2 }
      {#1}
  }
\cs_generate_variant:Nn \@@_step_aux:nN { e }
\cs_new:Npn \@@_trim_step:nN #1 #2
  {
%<@@=>
    \exp_args:Nee \@@silly@place@picture@@nn
      {
        \bool_if:NT #2
          {
            trim = \int_case:nn {#1}
              {
                {  1 } { 57 ~ 0 ~ 42 ~ 0 }
                {  2 } { 59 ~ 0 ~ 29 ~ 0 }
                {  3 } { 53 ~ 0 ~ 21 ~ 0 }
                {  4 } { 62 ~ 0 ~ 11 ~ 0 }
                {  5 } { 50 ~ 0 ~  8 ~ 0 }
                {  6 } { 33 ~ 0 ~ 27 ~ 0 }
                {  7 } { 45 ~ 0 ~ 13 ~ 0 }
                {  8 } { 56 ~ 0 ~ 27 ~ 0 }
                {  9 } { 38 ~ 0 ~ 12 ~ 0 }
                { 10 } {  7 ~ 0 ~ 43 ~ 0 }
                { 11 } { 33 ~ 0 ~ 33 ~ 0 }
                { 12 } { 57 ~ 0 ~  7 ~ 0 }
              }
            , clip
          }
      }
      { \int_eval:n {#1} }
  }
%    \end{macrocode}
% \end{macro}
%
% \begin{macro}[int]{\@@silly@place@picture@@nn}
%    \begin{macrocode}
\DeclareRobustCommand \@@silly@place@picture@@nn [2]
  {
    \includegraphics
      [ #1, page=#2, height=\silly@height ]
      { \silly@picture }
  }
\AtBeginDocument
  {
    \cs_if_exist_use:NF \pdfstringdefDisableCommands { \use_none:n }
      { \cs_set:Npn \@@silly@place@picture@@nn #1 #2 {#2} }
  }
\DeclareHookRule { begindocument }
  { sillypage } { after } { hyperref }
%    \end{macrocode}
% \end{macro}
%
%    \begin{macrocode}
%</package>
%    \end{macrocode}
%
% \end{implementation}
%
\endinput
